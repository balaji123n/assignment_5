
\documentclass{article}
\usepackage[utf8]{inputenc}
\usepackage{karnaugh-map}

% ADD TITLE HERE
\title{K-map for g}
\author{Balaji}
\date{7th January}

\begin{document}

\maketitle

\section{Boolean equation}
$g = B'C'D'+ ABCD'$
\section {Kmap for 2.5, g}
\begin{karnaugh-map}[4][4][1][][]
    \maxterms{2,3,4,5,6,8,9,10,11,12,13,14,15}
    \minterms{0,1,7}
    \implicantedge{0}{4}{2}{6}
    \implicantedge{0}{0}{8}{8}
    % note: position for start of \draw is (0, Y) where Y is
    % the Y size(number of cells high) in this case Y=2
    \draw[color=black, ultra thin] (0, 4) --
    node [pos=0.7, above right, anchor=south west] {$BA$} % YOU CAN CHANGE NAME OF VAR HERE, THE $X$ IS USED FOR ITALICS
    node [pos=0.7, below left, anchor=north east] {$DC$} % SAME FOR THIS
    ++(135:1);
        
    \end{karnaugh-map}
     
     \section{simplified equation} 
      $g = B'C'D'+ABCD'$
     \end{document}




